\documentclass[a4paper,12pt]{article}
\usepackage{amssymb,amsfonts,amsmath}
\usepackage[english]{babel}
\usepackage{latexsym}
\usepackage{epsfig}

% =======================================================================
% Margins --- save forests
\NeedsTeXFormat{LaTeX2e}
\oddsidemargin -10 pt       %   Left margin on odd-numbered pages.
\evensidemargin 10 pt       %   Left margin on even-numbered pages.
\marginparwidth 1 in        %   Width of marginal notes.
\oddsidemargin -1.5 true cm %   Note that \oddsidemargin = \evensidemargin
\evensidemargin -1.5 true cm
\marginparwidth 0.75 in
\textwidth 7.5 true in % Width of text line.
\textheight 26.0 true cm
\topmargin -2.5 true cm

% =======================================================================
% Environments problem, solution, remark

\newcount\probcnt
\newenvironment{problem}[1]{%
  \global\advance\probcnt1%
  \goodbreak\medskip\par\noindent\textbf{Problem~\the\probcnt%
    \if{#1}\empty\else~(#1)\fi.}~}%
{%
  \goodbreak
}

\newenvironment{solution}[1][]{%
  \goodbreak\smallskip\par\noindent\textbf{Solution{\if#1\empty\else~#1\fi}.}~}%
{%
  \goodbreak
}

\newenvironment{remark}[1][]{
  \goodbreak\smallskip\par
  \small
  \noindent\textbf{Remark{\if#1\empty\else~#1\fi}.}~%
}{%
  \goodbreak
  \normalsize
}

\newenvironment{lemma}[1][]{
  \goodbreak\smallskip\par
  \noindent\textit{Lemma{\if#1\empty\else~#1\fi}.}~%
}{%
  \goodbreak\smallskip
}

\newenvironment{proof}[1][]{
  \goodbreak\par
  \noindent\textit{Proof{\if#1\empty\else~#1\fi}.}~%
}{%
  \goodbreak\smallskip
}


% =======================================================================
%%% Some common macros
\newcommand{\RR}{{\mathbb{R}}}
\newcommand{\ZZ}{{\mathbb{Z}}}
\newcommand{\CC}{{\mathbb{C}}}
\newcommand{\QQ}{{\mathbb{Q}}}
\newcommand{\NN}{{\mathbb{N}}}
\newcommand{\tr}{\rm tr}
\newcommand{\ds}{\displaystyle}
\newcommand{\dx}{\mathrm{d}x}
\newcommand{\dy}{\mathrm{d}y}
\newcommand{\dz}{\mathrm{d}z}
\newcommand{\dt}{\mathrm{d}t}
\newcommand{\du}{\mathrm{d}u}
\newcommand{\GL}{\operatorname{GL}}
\newcommand{\rk}{\operatorname{rk}}

% == TITLES ==========================================================
\begin{document}
\begin{center}
  {\Large\textbf{Proposed problems for the IMC 2023}}
\end{center}

% =======================================================================
%%% DOCUMENT_BEGIN
% Do not change the document above this point
% =======================================================================

% =======================================================================
% My macros
% =======================================================================
% Insert your macros here

% \def\MyMacro{...}


% =======================================================================

\begin{problem}{G\'eza K\'os, E\"otv\"os University, Budapest}
  Tell me why and how to use this template.

  \begin{remark}
    You can add remarks to the problem statements as well.
  \end{remark}
\end{problem}

% -----------------------------------------------------------------------

\begin{solution}
  The aim of this template is to help merging and compiling the file of all
  problem proposals. In the past we received problems in various formats,
  using various LATeX packages and macros. The individual formats made it very
  hard to merge the problems into a single document.

  \begin{itemize}
  \item Please copy/rename this file and insert your problems. You can send
    more problems in one file.
    
  \item Please change only the region between the comments
    \texttt{DOCUMENT\_BEGIN} and \texttt{DOCUMENT\_END}. The lines outside
    this region will be ignored at merging.
    
  \item You can insert your macros in section ``My~macros''.

  \item Write the problem statements between
    \texttt{$\backslash$begin\{problem\}} and
    \texttt{$\backslash$end\{problem\}}. This environment expects your name
    and university as an argument.
    
  \item Similarly, the solutions can be put between
    \texttt{$\backslash$begin\{solution\}} and
    \texttt{$\backslash$end\{solution\}}. You can add an optional number to
    the solution in brackets.
    
  \item For lemmas and their proofs you can use the environments
    \texttt{lemma} and \texttt{proof}. In these cases as well, you can add an
    optional number in brackets.
    
  \item If you want to add remarks either to the statement or the solution,
    use the environment \texttt{remark}. You can optionally add a number in
    brackets.
    
    \begin{remark}
      Here is a remark.
    \end{remark}

    \begin{remark}[2]
      Here is a numbered remark.
    \end{remark}
    
    \begin{lemma}[3]
      \begin{equation}
        1+2=3
        \label{gkos:123}
      \end{equation}
    \end{lemma}
    
    \begin{proof}[5]
      To prove (\ref{gkos:123}), observe $1+2=2+1=3$.
    \end{proof}
    
  \item Choose your macro and label names in such a way that they will be
    probably different from the names used by other people. It is unlikely
    when several people use the label name \texttt{eq1} or the same macro
    name.
  \end{itemize}
  
\end{solution}

% =======================================================================

\begin{problem}{G\'eza K\'os, E\"otv\"os University, Budapest}
  Can I insert figures?
\end{problem}

% -----------------------------------------------------------------------

\begin{solution}[1 (using EPS)]
  Yes. Please use Encapsulated Postscript (EPS) figures and insert them using
  the \texttt{$\backslash$epsfig} command:
\begin{verbatim}
    \begin{center}
      \epsfig{figure=GKosElteCircles.eps, scale=1.0}
    \end{center}
\end{verbatim} 
  \begin{center}
    \epsfig{figure=GKosElteCircles.eps, scale=1.0}
  \end{center}
  Do not forget to send us the EPS files. :-)

  Similarly to the macro names and labels, do not use file names like
  ,,\texttt{fig1.eps}''. A good idea is to put some unique string into the
  file name, e.g. your name like ,,\texttt{GKosElteCircles.eps}''.
\end{solution}

% -----------------------------------------------------------------------

\begin{solution}[2 (without EPS)]
  If you have your figures in a different format and cannot convert them into
  EPS, send them anyway, we will convert them.
\end{solution}

% =======================================================================
% Do not change the document below this point
%%% DOCUMENT_END
% =======================================================================

\end{document}

