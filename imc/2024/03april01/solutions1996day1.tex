\documentclass{article}
\usepackage[utf8]{inputenc}
\usepackage[T1]{fontenc}
\usepackage{amsmath}
\usepackage{amsfonts}
\usepackage{amssymb}
\usepackage{enumitem}

\usepackage[margin=0.7in]{geometry} % Adjust the margins of document here.

\usepackage{tikz}                                          % Для простых рисунков в документе
\usetikzlibrary{matrix,arrows,decorations.pathmorphing,shapes.geometric,calc,snakes,backgrounds,arrows.meta}
\usepackage{xcolor}

\begin{document}
\pagestyle{plain}

\section*{Problems and solutions}

\subsection*{First day — August 2, 1996}

\textbf{Problem 1.} (10 points)\\
Let for $j = 0,\ldots,n$, $a_j = a_0 + jd$, where $a_0, d$ are fixed real numbers. Put
\[
A = \begin{pmatrix}
a_0 & a_1 & a_2 & \cdots & a_n \\
a_1 & a_0 & a_1 & \cdots & a_{n-1} \\
a_2 & a_1 & a_0 & \cdots & a_{n-2} \\
\vdots & \vdots & \vdots & \ddots & \vdots \\
a_n & a_{n-1} & a_{n-2} & \cdots & a_0 \\
\end{pmatrix}.
\]
Calculate $\det(A)$, where $\det(A)$ denotes the determinant of $A$.

\textbf{Solution.} Adding the first column of $A$ to the last column we get that
\[
\det(A) = (a_0 + a_n) \det \begin{pmatrix}
a_0 & a_1 & a_2 & \cdots & 1 \\
a_1 & a_0 & a_1 & \cdots & 1 \\
a_2 & a_1 & a_0 & \cdots & 1 \\
\vdots & \vdots & \vdots & \ddots & \vdots \\
a_n & a_{n-1} & a_{n-2} & \cdots & 1 \\
\end{pmatrix}.
\]
Subtracting the $n$-th row of the above matrix from the $(n+1)$-st one, $(n-1)$-st from $n$-th, $\ldots$, first from second we obtain that
\[
\det(A) = (a_0 + a_n) \det \begin{pmatrix}
a_0 & a_1 & a_2 & \cdots & 1 \\
d & -d & -d & \cdots & 0 \\
d & d & -d & \cdots & 0 \\
\vdots & \vdots & \vdots & \ddots & \vdots \\
d & d & d & \cdots & 0 \\
\end{pmatrix}.
\]
Hence,
\[
\det(A) = (-1)^n (a_0 + a_n) \det \begin{pmatrix}
d & -d & -d & \cdots & -d \\
d & d & -d & \cdots & -d \\
d & d & d & \cdots & -d \\
\vdots & \vdots & \vdots & \ddots & \vdots \\
d & d & d & \cdots & d \\
\end{pmatrix}.
\]

Adding the last row of the above matrix to the other rows we have
\[
\det(A) = (-1)^n (a_0+a_n) \det \begin{pmatrix}
2d & 0 & 0 & \cdots & 0 \\
2d & 2d & 0 & \cdots & 0 \\
2d & 2d & 2d & \cdots & 0 \\
\vdots & \vdots & \vdots & \ddots & \vdots \\
d & d & d & \cdots & d
\end{pmatrix} = (-1)^n (a_0+a_n)2^{n-1}d^n.
\]

\textbf{Problem 2.} (10 points)\\
Evaluate the definite integral
\[
\int_{-\pi}^{\pi} \frac{\sin nx}{(1 + 2^x)\sin x} \, dx,
\]
where \( n \) is a natural number.

\textbf{Solution.} We have
\[
I_n = \int_{-\pi}^{\pi} \frac{\sin nx}{(1 + 2^x)\sin x} \, dx
\]
\[
= \int_{0}^{\pi} \frac{\sin nx}{(1 + 2^x)\sin x} \, dx + \int_{-\pi}^{0} \frac{\sin nx}{(1 + 2^x)\sin x} \, dx.
\]
In the second integral we make the change of variable \( x = -x \) and obtain
\[
I_n = \int_{0}^{\pi} \frac{\sin nx}{(1 + 2^x)\sin x} \, dx + \int_{0}^{\pi} \frac{\sin nx}{(1 + 2^{-x})\sin x} \, dx
\]
\[
= \int_{0}^{\pi} \frac{(1 + 2^x)\sin nx}{(1 + 2^x)\sin x} \, dx
\]
\[
= \int_{0}^{\pi} \frac{\sin nx}{\sin x} \, dx.
\]

For \( n \geq 2 \) we have
\[
I_n - I_{n-2} = \int_{0}^{\pi} \frac{\sin nx - \sin(n - 2)x}{\sin x} \, dx
\]
\[
= 2 \int_{0}^{\pi} \cos(n - 1)x \, dx = 0.
\]

The answer
\[
I_n = \begin{cases}
0 & \text{if } n \text{ is even}, \\
\pi & \text{if } n \text{ is odd}.
\end{cases}
\]
follows from the above formula and \( I_0 = 0 \), \( I_1 = \pi \).

\textbf{Problem 3.} (15 points)
The linear operator \( A \) on the vector space \( V \) is called an involution if \( A^2 = E \) where \( E \) is the identity operator on \( V \). Let \( \dim V = n < \infty \).
\begin{enumerate}
\item[(i)] Prove that for every involution \( A \) on \( V \) there exists a basis of \( V \) consisting of eigenvectors of \( A \).
\item[(ii)] Find the maximal number of distinct pairwise commuting involutions on \( V \).
\end{enumerate}

\textbf{Solution.}
\begin{itemize}
    \item[(i)] Let $B = \frac{1}{2}(A + E)$. Then
    \[
    B^2 = \frac{1}{4}(A^2 + 2AE + E) = \frac{1}{4}(2AE + 2E) = \frac{1}{2}(A + E) = B.
    \]
    Hence $B$ is a projection. Thus there exists a basis of eigenvectors for $B$, and the matrix of $B$ in this basis is of the form $\text{diag}(1,\ldots, 1,0,\ldots, 0)$.

    Since $A = 2B - E$ the eigenvalues of $A$ are $\pm1$ only.

    \item[(ii)] Let $\{A_i : i \in I\}$ be a set of commuting diagonalizable operators on $V$, and let $A_1$ be one of these operators. Choose an eigenvalue $\lambda$ of $A_1$ and denote $V_{\lambda} = \{v \in V : A_1v = \lambda v\}$. Then $V_{\lambda}$ is a subspace of $V$, and since $A_1A_i = A_iA_1$ for each $i \in I$ we obtain that $V_{\lambda}$ is invariant under each $A_i$. If $V_{\lambda} = V$ then $A_1$ is either $E$ or $-E$, and we can start with another operator $A_i$. If $V_{\lambda} \neq V$ we proceed by induction on $\dim V$ in order to find a common eigenvector for all $A_i$. Therefore $\{A_i : i \in I\}$ are simultaneously diagonalizable.

    If they are involutions then $|I| \leq 2^n$ since the diagonal entries may equal $1$ or $-1$ only.
\end{itemize}

\textbf{Problem 4.} (15 points)\\
Let $a_1 = 1$, $a_n = \frac{1}{n} \sum_{k=1}^{n-1} a_k a_{n-k}$ for $n \geq 2$. Show that
\begin{itemize}
    \item[(i)] $\limsup_{n \to \infty} |a_n|^{1/n} < 2^{-1/2}$,
    \item[(ii)] $\limsup_{n \to \infty} |a_n|^{1/n} \geq 2/3$.
\end{itemize}

\textbf{Solution.}
\begin{itemize}
    \item[(i)] We show by induction that
    \[
    (*) \quad a_n \leq q^n \text{ for } n \geq 3,
    \]

    where $q = 0.7$ and use that $0.7 < 2^{-1/2}$. One has $a_1 = 1, a_2 = \frac{1}{2}, a_3 = \frac{1}{3}$,
    \[
    a_4 = \frac{11}{48}.
    \]
    Therefore $(*)$ is true for $n = 3$ and $n = 4$. Assume $(*)$ is true for $n \leq N - 1$ for some $N \geq 5$. Then
    \[
    a_N = \frac{2}{N} a_{N-1} + \frac{1}{N} a_{N-2} + \frac{1}{N} \sum_{k=3}^{N-3} a_k a_{N-k} \leq \frac{2}{N} q^{N-1} + \frac{1}{N} q^{N-2} + \frac{N-5}{N} q^N \leq q^N
    \]
    because $\frac{2}{q} + \frac{1}{q^2} \leq 5$.

    \item[(ii)] We show by induction that
    \[
    a_n \geq q^n \text{ for } n \geq 2,
    \]
    where $q = \frac{2}{3}$. One has $a_2 = \frac{1}{2} > (\frac{2}{3})^2 = q^2$. Going by induction we have
    \[
    a_N = \frac{2}{N} a_{N-1} + \frac{1}{N} \sum_{k=2}^{N-2} a_k a_{N-k} \geq \frac{2}{N} q^{N-1} + \frac{N-3}{N} q^N = q^N
    \]
    because $\frac{2}{q} = 3$.
\end{itemize}

\textbf{Problem 5.} (25 points)
\begin{itemize}
    \item[(i)] Let \( a, b \) be real numbers such that \( b \leq 0 \) and \( 1 + ax + bx^2 \geq 0 \) for every \( x \) in \( [0,1] \). Prove that
    \[
    \lim_{n \to \infty} n \int_{0}^{1} (1 + ax + bx^2)^n \, dx =
    \begin{cases}
    -\frac{1}{a} & \text{if } a < 0, \\
    +\infty & \text{if } a \geq 0.
    \end{cases}
    \]

    \item[(ii)] Let \( f : [0,1] \rightarrow [0,\infty) \) be a function with a continuous second derivative and let \( f''(x) \leq 0 \) for every \( x \) in \( [0,1] \). Suppose that \( L = \lim_{n \to \infty} n \int_{0}^{1} (f(x))^n \, dx \) exists and \( 0 < L < +\infty \). Prove that \( f' \) has a constant sign and \( \min_{x \in [0,1]} |f'(x)| = L^{-1} \).
\end{itemize}

\textbf{Solution.}
\begin{itemize}
    \item[(i)] With a linear change of the variable (i) is equivalent to:
    (i') Let \( a, b, A \) be real numbers such that \( b \leq 0 \), \( A > 0 \) and \( 1+ax+bx^2 > 0 \) for every \( x \) in \( [0,A] \). Denote \( I_n = n \int_{0}^{A} (1 + ax + bx^2)^n \, dx \). Prove that
    \[
    \lim_{n \to \infty} I_n = -\frac{1}{a} \text{ when } a < 0 \text{ and } \lim_{n \to \infty} I_n = +\infty \text{ when } a \geq 0.
    \]
    Let \( a < 0 \). Set \( f(x) = e^{ax} - (1 + ax + bx^2) \). Using that \( f(0) = f'(0) = 0 \) and \( f''(x) = a^2e^{ax} - 2b \) we get for \( x > 0 \) that
    \[
    0 < e^{ax} - (1 + ax + bx^2) < cx^2
    \]
    where \( c = \frac{a^2}{2} - b \). Using the mean value theorem we get
    \[
    0 < e^{anx} - (1 + ax + bx^2)^n < cx^2ne^{a(n-1)x}.
    \]
    Therefore
    \[
    0 < n \int_{0}^{A} e^{anx}\,dx - n \int_{0}^{A} (1 + ax + bx^2)^n\,dx < cn^2 \int_{0}^{A} x^2e^{a(n-1)x}\,dx.
    \]
    Using that
    \[
    n \int_{0}^{A} e^{anx}\,dx = \frac{e^{anA} - 1}{a} \rightarrow \frac{-1}{a} \quad \text{as } n \to \infty
    \]
    and
    \[
    \int_{0}^{A} x^2e^{a(n-1)x}\,dx < \frac{1}{\lvert a \rvert^3(n - 1)^3} \int_{0}^{\infty} t^2e^{-t}\,dt
    \]
    we get \( (i') \) in the case \( a < 0 \).

    Let \( a \geq 0 \). Then for \( n > \max\{A^{-2}, -b\} - 1 \) we have
    \[
    n \int_{0}^{A} (1 + ax + bx^2)^n \, dx > n \int_{0}^{\frac{1}{\sqrt{n+1}}} (1 + bx^2)^n \, dx
    \]
    \[
    > n \cdot \frac{1}{\sqrt{n+1}} \cdot \left(1 + \frac{b}{n+1}\right)^n
    \]
    \[
    > \frac{n}{\sqrt{n+1}} e^b \rightarrow \infty \quad \text{as } n \to \infty.
    \]
    (i) is proved.

    \item[(ii)] Denote \( I_n = n \int_{0}^{1} (f(x))^n \, dx \)
    and \( M = \max\limits_{x \in [0,1]} f(x) \).

    For \( M < 1 \) we have \( I_n \leq nM^n \rightarrow 0 \), a contradiction.

    If \( M > 1 \) since \( f \) is continuous there exists an interval
    \( I \subset [0,1] \) with \( |I| > 0 \) such that \( f(x) > 1 \)
    for every \( x \in I \). Then \( I_n \geq n|I| \rightarrow \infty \),
    a contradiction. Hence \( M = 1 \). Now we prove that \( f' \) has a constant
    sign. Assume the opposite. Then \( f'(x_0) = 0 \) for some \( x \in (0,1) \).
    Then \( f(x_0) = M = 1 \) because \( f'' \leq 0 \).
    For \( x_0 + h \) in \([0,1]\), \( f(x_0 + h) = 1 + \frac{h^2}{2} f''(\xi) \),
    for some \( \xi \) in \( (x_0, x_0 + h) \).
    Let \( m = \min_{x \in [0,1]} f''(x) \). So, \( f(x_0 + h) \geq 1 + \frac{h^2}{2} m \).

    Let \( \delta > 0 \) be such that \( 1 + \frac{\delta^2}{2} m > 0 \) and \( x_0 + \delta < 1 \).
    Then
    \[
    I_n \geq n \int_{x_0}^{x_0+\delta} (f(x))^n \, dx \geq n \int_{0}^{\delta} \left( 1 + \frac{m}{2} h^2 \right)^n \, dh \rightarrow \infty \quad \text{as } n \to \infty
    \]
    in view of \( (i') \) — a contradiction. Hence \( f \) is monotone and \( M = f(0) \) or \( M = f(1) \).

    Let \( M = f(0) = 1 \). For \( h \) in \([0,1]\)
    \[
    1 + h f'(0) \geq f(h) \geq 1 + h f'(0) + \frac{m}{2} h^2,
    \]
    where \( f'(0) \neq 0 \), because otherwise we get a contradiction as above.
    Since \( f(0) = M \) the function \( f \) is decreasing and hence \( f'(0) < 0 \).
    Let \( 0 < A < 1 \) be such that \( 1 + A f'(0) + \frac{m}{2} A^2 > 0 \).
    Then
    \[
    n \int_{0}^{A} (1 + h f'(0))^n \, dh \geq n \int_{0}^{A} (f(x))^n \, dx \geq n \int_{0}^{A} \left(1 + h f'(0) + \frac{m}{2} h^2 \right)^n \, dh.
    \]
    From \( (i') \) the first and the third integral tend to \( -\frac{1}{f'(0)} \) as \( n \to \infty \), hence so does the second.
    Also
    \[
    n \int_{A}^{1} (f(x))^n \, dx \leq n (f(A))^n \rightarrow 0 \quad (f(A) < 1).
    \]
    We get \( L = -\frac{1}{f'(0)} \) in this case.

    If \( M = f(1) \) we get in a similar way \( L = -\frac{1}{f'(1)} \).
\end{itemize}

\textbf{Problem 6.} (25 points)
    Upper content of a subset \( E \) of the plane \( \mathbb{R}^2 \) is defined as
    \[
    C(E) = \inf\left\{ \sum_{i=1}^{n} \text{diam}(E_i) \right\}
    \]
    where inf is taken over all finite families of sets \( E_1, \ldots, E_n, n \in \mathbb{N} \), in \( \mathbb{R}^2 \) such that \( E \subseteq \bigcup_{i=1}^{n} E_i \).
Lower content of \( E \) is defined as
\[
K(E) = \sup\{\text{length}(L) : L \text{ is a closed line segment onto which } E \text{ can be contracted}\}.
\]

Show that
\begin{itemize}
    \item[(a)] \( C(L) = \text{length}(L) \) if \( L \) is a closed line segment;
    \item[(b)] \( C(E) \geq K(E) \);
    \item[(c)] the equality in (b) needs not hold even if \( E \) is compact.
\end{itemize}
\textbf{Hint.} If \( E = T \cup T' \) where \( T \) is the triangle with vertices \( (-2, 2), (2, 2) \) and \( (0, 4) \), and \( T' \) is its reflexion about the \( x \)-axis, then \( C(E) = 8 > K(E) \).

\textbf{Remarks:} All \textit{distances} used in this problem are Euclidean. \textit{Diameter} of a set \( E \) is \( \text{diam}(E) = \sup\{\text{dist}(x,y) : x, y \in E\} \). \textit{Contraction} of a set \( E \) to a set \( F \) is a mapping \( f : E \to F \) such that \( \text{dist}(f(x), f(y)) \leq \text{dist}(x, y) \) for all \( x, y \in E \). A set \( E \) can be contracted onto a set \( F \) if there is a contraction \( f \) of \( E \) to \( F \) which is onto, i.e., such that \( f(E) = F \). \textit{Triangle} is defined as the union of the three segments joining its vertices, i.e., it does not contain the interior.

\textbf{Solution.}
\begin{itemize}
    \item[(a)] The choice \( E_1 = L \) gives \( C(L) \leq \text{length}(L) \). If \( E \subseteq \bigcup_{i=1}^{n} E_i \), then
    \[
    \sum_{i=1}^{n} \text{diam}(E_i) \geq \text{length}(L).
    \]
    By induction, \( n=1 \) obvious, and assuming that \( E_{n+1} \) contains the end point \( a \) of \( L \), define the segment \( L_{\varepsilon} = \{x \in L : \text{dist}(x, a) > \text{diam}(E_{n+1})+\varepsilon\} \) and use induction assumption to get
    \[
    \sum_{i=1}^{n+1} \text{diam}(E_i) \geq \text{length}(L_{\varepsilon}) + \text{diam}(E_{n+1}) \geq \text{length}(L) - \varepsilon;
    \]
    but \( \varepsilon > 0 \) is arbitrary.

    \item[(b)] If \( f \) is a contraction of \( E \) onto \( L \) and \( E \subseteq \bigcup_{i=1}^{n} E_i \), then \( L \subseteq \bigcup_{i=1}^{n} f(E_i) \) and
    \[
    \text{length}(L) \leq \sum_{i=1}^{n} \text{diam}(f(E_i)) \leq \sum_{i=1}^{n} \text{diam}(E_i).
    \]

    \item[(c1)] Let \( E = T \cup T' \) where \( T \) is the triangle with vertices \( (-2, 2), (2, 2) \) and \( (0, 4) \), and \( T' \) is its reflexion about the \( x \)-axis. Suppose \( E \subseteq \bigcup_{i=1}^{n} E_i \). If no set among \( E_i \) meets both \( T \) and \( T' \), then \( E_i \) may be partitioned into covers of segments \( [(-2, 2), (2, 2)] \) and \( [(-2, -2), (2, -2)] \), both of length 4, so
    \[
    \sum_{i=1}^{n} \text{diam}(E_i) \geq 8.
    \]
    If at least one set among \( E_i \), say \( E_k \), meets both \( T \) and \( T' \), choose \( a \in E_k \cap T \) and \( b \in E_k \cap T' \) and note that the sets \( E'_i = E_i \) for \( i \neq k \), \( E'_k = E_k \cup [a,b] \) cover \( T \cup T' \cup [a,b] \), which is a set of upper content
    at least 8, since its orthogonal projection onto y-axis is a segment of length 8. Since \(\text{diam}(E_j) = \text{diam}(E_j')\), we get
    \[
    \sum_{i=1}^{n} \text{diam}(E_i) \geq 8.
    \]
    (c2) Let \( f \) be a contraction of \( E \) onto \( L = [a', b'] \). Choose \( a = (a_1, a_2) \), \( b = (b_1, b_2) \in E \) such that \( f(a) = a' \) and \( f(b) = b' \). Since \(\text{length}(L) = \text{dist}(a', b') \leq \text{dist}(a, b) \) and since the triangles have diameter only 4, we may assume that \( a \in T \) and \( b \in T' \). Observe that if \( a_2 \leq 3 \) then \( a \) lies on one of the segments joining some of the points \((-2,2), (2,2), (-1,3), (1,3)\); since all these points have distances from vertices, and so from points, of \( T_2 \) at most \( \sqrt{50} \), we get that \(\text{length}(L) \leq \text{dist}(a, b) \leq \sqrt{50}\). Similarly, if \( b_2 \geq -3 \). Finally, if \( a_2 > 3 \) and \( b_2 < -3 \), we note that every vertex, and so every point of \( T \) is in the distance at most \( \sqrt{10} \) for \( a \) and every vertex, and so every point, of \( T' \) is in the distance at most \( \sqrt{10} \) of \( b \). Since \( f \) is a contraction, the image of \( T \) lies in a segment containing \( a' \) of length at most \( \sqrt{10} \) and the image of \( T' \) lies in a segment containing \( b' \) of length at most \( \sqrt{10} \). Since the union of these two images is \( L \), we get \(\text{length}(L) \leq 2\sqrt{10} \leq \sqrt{50}\). Thus \( K(E) \leq \sqrt{50} < 8 \).
\end{itemize}

\end{document}
