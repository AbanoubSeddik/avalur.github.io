\documentclass{article}
\usepackage[utf8]{inputenc}
\usepackage[T1]{fontenc}
\usepackage{amsmath}
\usepackage{amsfonts}
\usepackage{amssymb}
\usepackage{enumitem}

\usepackage[margin=0.7in]{geometry} % Adjust the margins of document here.

\usepackage{tikz}                                          % Для простых рисунков в документе
\usetikzlibrary{matrix,arrows,decorations.pathmorphing,shapes.geometric,calc,snakes,backgrounds,arrows.meta}
\usepackage{xcolor}

\begin{document}
\pagestyle{plain}

\section*{Problems}

\subsection*{First day — August 2, 1996}

\textbf{Problem 1.} (10 points)\\
Let for $j = 0,\ldots,n$, $a_j = a_0 + jd$, where $a_0, d$ are fixed real numbers. Put
\[
A = \begin{pmatrix}
a_0 & a_1 & a_2 & \cdots & a_n \\
a_1 & a_0 & a_1 & \cdots & a_{n-1} \\
a_2 & a_1 & a_0 & \cdots & a_{n-2} \\
\vdots & \vdots & \vdots & \ddots & \vdots \\
a_n & a_{n-1} & a_{n-2} & \cdots & a_0 \\
\end{pmatrix}.
\]
Calculate $\det(A)$, where $\det(A)$ denotes the determinant of $A$.

\textbf{Problem 2.} (10 points)\\
Evaluate the definite integral
\[
\int_{-\pi}^{\pi} \frac{\sin nx}{(1 + 2^x)\sin x} \, dx,
\]
where \( n \) is a natural number.

\textbf{Problem 3.} (15 points)
The linear operator \( A \) on the vector space \( V \) is called an involution if \( A^2 = E \) where \( E \) is the identity operator on \( V \). Let \( \dim V = n < \infty \).
\begin{enumerate}
\item[(i)] Prove that for every involution \( A \) on \( V \) there exists a basis of \( V \) consisting of eigenvectors of \( A \).
\item[(ii)] Find the maximal number of distinct pairwise commuting involutions on \( V \).
\end{enumerate}

\textbf{Problem 4.} (15 points)\\
Let $a_1 = 1$, $a_n = \frac{1}{n} \sum_{k=1}^{n-1} a_k a_{n-k}$ for $n \geq 2$. Show that
\begin{itemize}
    \item[(i)] $\limsup_{n \to \infty} |a_n|^{1/n} < 2^{-1/2}$,
    \item[(ii)] $\limsup_{n \to \infty} |a_n|^{1/n} \geq 2/3$.
\end{itemize}

\textbf{Problem 5.} (25 points)
\begin{itemize}
    \item[(i)] Let \( a, b \) be real numbers such that \( b \leq 0 \) and \( 1 + ax + bx^2 \geq 0 \) for every \( x \) in \( [0,1] \). Prove that
    \[
    \lim_{n \to \infty} n \int_{0}^{1} (1 + ax + bx^2)^n \, dx =
    \begin{cases}
    -\frac{1}{a} & \text{if } a < 0, \\
    +\infty & \text{if } a \geq 0.
    \end{cases}
    \]

    \item[(ii)] Let \( f : [0,1] \rightarrow [0,\infty) \) be a function with a continuous second derivative and let \( f''(x) \leq 0 \) for every \( x \) in \( [0,1] \). Suppose that \( L = \lim_{n \to \infty} n \int_{0}^{1} (f(x))^n \, dx \) exists and \( 0 < L < +\infty \). Prove that \( f' \) has a constant sign and \( \min_{x \in [0,1]} |f'(x)| = L^{-1} \).
\end{itemize}

\textbf{Problem 6.} (25 points)
    Upper content of a subset \( E \) of the plane \( \mathbb{R}^2 \) is defined as
    \[
    C(E) = \inf\left\{ \sum_{i=1}^{n} \text{diam}(E_i) \right\}
    \]
    where inf is taken over all finite families of sets \( E_1, \ldots, E_n, n \in \mathbb{N} \), in \( \mathbb{R}^2 \) such that \( E \subseteq \bigcup_{i=1}^{n} E_i \).
Lower content of \( E \) is defined as
\[
K(E) = \sup\{\text{length}(L) : L \text{ is a closed line segment onto which } E \text{ can be contracted}\}.
\]

Show that
\begin{itemize}
    \item[(a)] \( C(L) = \text{length}(L) \) if \( L \) is a closed line segment;
    \item[(b)] \( C(E) \geq K(E) \);
    \item[(c)] the equality in (b) needs not hold even if \( E \) is compact.
\end{itemize}
\textbf{Hint.} If \( E = T \cup T' \) where \( T \) is the triangle with vertices \( (-2, 2), (2, 2) \) and \( (0, 4) \), and \( T' \) is its reflexion about the \( x \)-axis, then \( C(E) = 8 > K(E) \).

\textbf{Remarks:} All \textit{distances} used in this problem are Euclidean.
\textit{Diameter} of a set \( E \) is \( \text{diam}(E) = \sup\{\text{dist}(x,y): x, y \in E\} \).
\textit{Contraction} of a set \( E \) to a set \( F \) is a mapping \( f : E \to F \) such that \( \text{dist}(f(x), f(y)) \leq \text{dist}(x, y) \)
for all \( x, y \in E \). A set \( E \) can be contracted onto a set \( F \) if there is a contraction \( f \) of \( E \) to \( F \) which is onto,
i.e., such that \( f(E) = F \).
\textit{Triangle} is defined as the union of the three segments joining its vertices, i.e., it does not contain the interior.

\end{document}
