\documentclass{article}
\usepackage[utf8]{inputenc}
\usepackage[T1]{fontenc}
\usepackage{amsmath}
\usepackage{amsfonts}
\usepackage{amssymb}
\usepackage{enumitem}

\usepackage[margin=0.7in]{geometry} % Adjust the margins of document here.

\usepackage{tikz}                                          % Для простых рисунков в документе
\usetikzlibrary{matrix,arrows,decorations.pathmorphing,shapes.geometric,calc,snakes,backgrounds,arrows.meta}
\usepackage{xcolor}

\begin{document}
\pagestyle{plain}

\section*{Problems and solutions}

\subsection*{First day — August 3, 1998}

\textbf{Problem 1.} (20 points)
Let $V$ be a 10-dimensional real vector space and $U_1$ and $U_2$ two linear subspaces
such that $U_1 \subseteq U_2$, $\dim{U_1} = 3$ and $\dim{U_2} = 6$.
Let $\mathcal{E}$ be the set of all linear maps $T : V \rightarrow V$
which have $U_1$ and $U_2$ as invariant subspaces (i.e., $T(U_1) \subseteq U_1$
and $T(U_2) \subseteq U_2$). Calculate the dimension of $\mathcal{E}$ as a real vector space.

\textbf{Solution} First choose a basis $\{v_1, v_2, v_3\}$ of $U_1$.
It is possible to extend this basis with vectors $v_4,v_5$ and $v_6$
to get a basis of $U_2$. In the same way we can extend a basis
of $U_2$ with vectors $v_7, \ldots, v_{10}$ to get as basis of $V$.

Let $T \in \mathcal{E}$ be an endomorphism
which has $U_1$ and $U_2$ as invariant subspaces.
Then its matrix, relative to the basis $\{v_1, \ldots, v_{10}\}$ is of the form
\[
\begin{bmatrix}
* & * & * & * & * & * & * & * & * & * \\
* & * & * & * & * & * & * & * & * & * \\
* & * & * & * & * & * & * & * & * & * \\
0 & 0 & 0 & * & * & * & * & * & * & * \\
0 & 0 & 0 & * & * & * & * & * & * & * \\
0 & 0 & 0 & * & * & * & * & * & * & * \\
0 & 0 & 0 & 0 & 0 & 0 & * & * & * & * \\
0 & 0 & 0 & 0 & 0 & 0 & * & * & * & * \\
0 & 0 & 0 & 0 & 0 & 0 & * & * & * & * \\
0 & 0 & 0 & 0 & 0 & 0 & * & * & * & * \\
\end{bmatrix}.
\]
So the dimension of $\mathcal{E}$ is the number of stars above: $9+18+40 = 67$.

\textbf{Problem 2.}
Prove that the following proposition holds for \( n = 3 \) (5 points)
and \( n = 5 \) (7 points), and does not hold for \( n = 4 \) (8 points).

``For any permutation \(\pi_1\) of \(\{1,2,\ldots,n\}\)
different from the identity there is a permutation \(\pi_2\)
such that any permutation \(\pi\) can be obtained from \(\pi_1\) and \(\pi_2\)
using only compositions (for example, \(\pi = \pi_1 \circ \pi_1 \circ \pi_2 \circ \pi_1\)).''

\textbf{Solution}
Let \( S_n \) be the group of permutations of \(\{1,2,\ldots,n\}\).
\begin{enumerate}
    \item When \( n = 3 \) the proposition is obvious: if \( x = (12) \) we choose \( y = (123) \); if \( x = (123) \) we choose \( y = (12) \).
    \item \( n = 4 \). Let \( x = (12)(34) \). Assume that there exists \( y \in S_n \), such that \( S_4 = \langle x, y \rangle \). Denote by \( K \) the invariant subgroup
    \[ K = \{\text{id}, (12)(34), (13)(24), (14)(23)\}. \]
    By the fact that \( x \) and \( y \) generate the whole group \( S_4 \), it follows that the factor group \( S_4 / K \) contains only powers of \( \tilde{y} = yK \), i.e., \( S_4 / K \) is cyclic. It is easy to see that this factor-group is not commutative (something more this group is not isomorphic to \( S_3 \)).
    \item \( n = 5 \)
    \begin{enumerate}
        \item If \( x = (12) \), then for \( y \) we can take \( y = (12345) \).
        \item If \( x = (123) \), we set \( y = (124)(35) \). Then \( y^3x y^3 = (125) \) and \( y^4 = (124) \). Therefore \( (123) \), \( (124) \), \( (125) \) \(\in \langle x, y \rangle\)- the subgroup generated by \( x \) and \( y \). From the fact that \( (123) \), \( (124) \), \( (125) \) generate the alternating subgroup \( A_5 \), it follows that \( A_5 \subseteq \langle x, y \rangle \). Moreover \( y \) is an odd permutation, hence \( \langle x, y \rangle = S_5 \).
        \item If \( x = (123)(45) \), then as in b) we see that for \( y \) we can take the element \( (124) \).
        \item If \( x = (1234) \), we set \( y = (12345) \). Then \( (yx)^3 = (24) \in \langle x, y \rangle \), \( x^2(24) = (13) \in \langle x, y \rangle \) and \( y^2 = (13524) \in \langle x, y \rangle \). By the fact \( (13) \in \langle x, y \rangle \) and \( (13524) \in \langle x, y \rangle \), it follows that \( \langle x, y \rangle = S_5 \).
        \item If \( x = (12)(34) \), then for \( y \) we can take \( y = (1354) \). Then \( y^2x = (125) \), \( y^3x = (124)(53) \) and by \( c \) \( S_5 = \langle x, y \rangle \).
        \item If \( x = (12345) \), then it is clear that for \( y \) we can take the element \( y = (12) \).
    \end{enumerate}
\end{enumerate}

\textbf{Problem 3.} Let \( f(x) = 2x(1 - x) \), \( x \in \mathbb{R} \). Define
\[
f_n = \underbrace{f \circ \ldots \circ f}_{n \text{ times}}.
\]

\begin{enumerate}
    \item[a)] (10 points) Find \(\displaystyle\lim_{n\to\infty} \int_{0}^{1} f_n(x) \, dx\).
    \item[b)] (10 points) Compute \(\int_{0}^{1} f_n(x) \, dx\) for \( n = 1,2,\ldots \).
\end{enumerate}

\noindent \textbf{Solution.}
\begin{enumerate}
    \item[a)] Fix \( x = x_0 \in (0,1) \). If we denote \( x_n = f_n(x_0) \), \( n = 1,2,\ldots \) it is easy to see that \( x_1 \in (0,1/2] \), \( x_1 \leq f(x_1) \leq 1/2 \) and \( x_n \leq f(x_n) \leq 1/2 \) (by induction). Then \( (x_n) \) is a bounded non-decreasing sequence and, since \( x_{n+1} = 2x_n(1 - x_n) \), the limit \( l = \lim_{n\to\infty} x_n \) satisfies \( l = 2l(1 - l) \), which implies \( l = 1/2 \). Now the monotone convergence theorem implies that
    \[
    \lim_{n\to\infty} \int_{0}^{1} f_n(x) \, dx = 1/2.
    \]
    \item[b)] We prove by induction that
\end{enumerate}
\[
f_n(x) = \frac{1}{2} - 2^{2n-1} \left( x - \frac{1}{2} \right)^{2^n}
\]
holds for \( n = 1,2,\ldots \).
For \( n = 1 \) this is true,
since \( f(x) = 2x(1 - x) = \frac{1}{2} - 2(x - \frac{1}{2})^2 \).
If it holds for some \( n = k \), then we have

\[
f_{k+1}(x) = f_k(f(x)) = \frac{1}{2} - 2^{2k-1}\left(\frac{1}{2} - 2\left(x - \frac{1}{2}\right)^2\right)^{2^k} = \frac{1}{2} - 2^{2k-1}\left(-2(x - \frac{1}{2})^2\right)^{2^k} = \frac{1}{2} - 2^{2k+1}\left(x - \frac{1}{2}\right)^{2^{k+1}},
\]

which is our statement for \( n = k + 1 \).

Using it we can compute the integral,
\[
\int_{0}^{1} f_n(x) \, dx = \left[ \frac{x}{2} - \frac{2^{2n-1}}{2n+1}\left(x - \frac{1}{2}\right)^{2n+1} \right]_{x=0}^{x=1} = \frac{1}{2} - \frac{1}{2(2n+1)}.
\]

\textbf{Problem 4.} (20 points) The function \( f : \mathbb{R} \rightarrow \mathbb{R} \) is twice differentiable and satisfies \( f(0) = 2, f'(0) = -2 \) and \( f(1) = 1 \). Prove that there exists a real number \( \xi \in (0, 1) \) for which
\[ f(\xi) \cdot f'(\xi) + f''(\xi) = 0. \]

\textbf{Solution.} Define the function
\[ g(x) = \frac{1}{2} f^2(x) + f'(x). \]
Because \( g(0) = 0 \) and
\[ f(x) \cdot f'(x) + f''(x) = g'(x), \]
it is enough to prove that there exists a real number \( 0 < \eta \leq 1 \) for which \( g(\eta) = 0 \).

a) If \( f \) is never zero, let
\[ h(x) = \frac{x}{2} - \frac{1}{f(x)}. \]
Because \( h(0) = h(1) = -\frac{1}{2} \), there exists a real number \( 0 < \eta < 1 \) for which \( h'(\eta) = 0 \). But \( g = f^2 \cdot h' \), and we are done.

b) If \( f \) has at least one zero, let \( z_1 \) be the first one and \( z_2 \) be the last one. (The set of the zeros is closed.) By the conditions, \( 0 < z_1 \leq z_2 < 1 \).

The function \( f \) is positive on the intervals \( [0, z_1) \) and \( (z_2, 1] \); this implies that \( f'(z_1) \leq 0 \) and \( f'(z_2) \geq 0 \). Then \( g(z_1) = f'(z_1) \leq 0 \) and \( g(z_2) = f'(z_2) \geq 0 \), and there exists a real number \( \eta \in [z_1, z_2] \) for which \( g(\eta) = 0 \).

\textbf{Remark.} For the function \( f(x) = \frac{2}{x+1} \) the conditions hold and \( f \cdot f' + f'' \) is constantly 0.

\textbf{Problem 5.}
Let \( P \) be an algebraic polynomial of degree \( n \) having only real zeros and real coefficients.

\begin{enumerate}
    \item[(a)] (15 points) Prove that for every real \( x \) the following inequality holds:
    \begin{equation}
        (n - 1)\left({P'(x)}\right)^2 \geq n{P(x)}{P''(x)}
    \end{equation}
    \item[(b)] (5 points) Examine the cases of equality.
\end{enumerate}

\textbf{Solution.} Observe that both sides of (1) are identically equal to zero
if \( n = 1 \). Suppose that \( n > 1 \).
Let \( x_1, \ldots, x_n \) be the zeros of \( P \). Clearly (1) is true when
\( x = x_i \), \( i \in \{1, \ldots, n\} \), and equality is possible only if \( P'(x_i) = 0 \), i.e., if \( x_i \) is a multiple zero of \( P \). Now suppose that \( x \) is not a zero of \( P \). Using the identities
\begin{align*}
    \frac{P'(x)}{P(x)} &= \sum_{i=1}^{n} \frac{1}{x - x_i}, \\
    \frac{P''(x)}{P(x)} &= \sum_{1 \leq i < j \leq n} \frac{2}{(x - x_i)(x - x_j)},
\end{align*}
we find
\begin{equation*}
    (n - 1)\left(\frac{P'(x)}{P(x)}\right)^2 - n\frac{P''(x)}{P(x)} =
\sum_{i=1}^{n} \left(\frac{n-1}{(x - x_i)^2}\right)^2 -
\sum_{1 \leq i < j \leq n} \frac{2}{(x - x_i)(x - x_j)}.
\end{equation*}
But this last expression is simlply
\begin{equation*}
    \sum_{1 \leq i < j \leq n} \left(\frac{1}{(x - x_i)} - \frac{1}{(x - x_j)} \right)^2,
\end{equation*}
and therefore is positive. The inequality is proved.
In order that (1) holds with equality sign for every real \( x \)
it is necessary that \( x_1 = x_2 = \ldots = x_n \).
A direct verification shows that indeed, if \( P(x) = c(x - x_1)^n \), then (1) becomes an identity.

\textbf{Problem 6.} Let \( f : [0,1] \rightarrow \mathbb{R} \)
be a continuous function with the property that for any \( x \) and \( y \) in the interval,
\[
xf(y)+yf(x) \leq 1.
\]

a) (15 points) Show that
\[
\int_{0}^{1} f(x) \, dx \leq \frac{\pi}{4}.
\]

b) (5 points) Find a function, satisfying the condition, for which there is equality.

\textbf{Solution} Observe that the integral is equal to
\[
\int_{0}^{\frac{\pi}{2}} f(\sin \theta) \cos \theta \, d\theta
\]
and to
\[
\int_{0}^{\frac{\pi}{2}} f(\cos \theta) \sin \theta \, d\theta.
\]

So, twice the integral is at most
\[
\int_{0}^{\frac{\pi}{2}} 1 \, d\theta = \frac{\pi}{2}.
\]

Now let \( f(x) = \sqrt{1 - x^2} \). If \( x = \sin \theta \) and \( y = \sin \phi \) then
\[
x f(y) + y f(x) = \sin \theta \cos \phi + \sin \phi \cos \theta = \sin(\theta + \phi) \leq 1.
\]

\end{document}
